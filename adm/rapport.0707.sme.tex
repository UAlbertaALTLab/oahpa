\documentclass[a4paper,norsk]{article}
\usepackage{babel}
\usepackage{ucs}
\usepackage[utf8x]{inputenc}
\usepackage[T1]{fontenc}
\usepackage{a4wide}

\usepackage{hyperref}
\usepackage{graphics}
\usepackage{url}

\begin{document}


\title{Interaktiivalaš sámegiel arena interneahtas: Raporta, suoidnemánu 2007}


\author{Trond Trosterud ja Lene Antonsen\\
Humanisttalaš fakultehta\\
Romssa Universitehta}

\maketitle


Prošeavttas \textit{Interaktiivalaš sámegiel arena interneahtas - hárjehallanbáiki álggahalliide ja daidda geat juo máhttet olu} leat guokte váldoulbmila:

\begin{enumerate}
\item Ráhkadit prográmmaid vai sáhttá interaktiivvalaččat hárjehallat gulahallangálggaid sámegielas.
\item Ráhkadit interaktiivvalaš spealuid sámegillii go viiddidit sámegielat cealkkavuorkká mii lea interaktiivvalaš speallanvuođus \url{http://visl.sdu.dk}
\end{enumerate}

\section{Bargu dán rádjái}

\subsection{Interaktiivvalaš pedagogalaš spealut}

Álgoplánema maŋŋil 2007 giđa de prošeakta bođii albma ládje  johtui easkka 1.6.2007 dalle go prošeavtta pedagoga sáhtii bargagoahtit. Nu guhkás leat programmerár, pedagoga ja datalingvista ovttas ráhkadan guokte dokumeantta: Nuppis leat čilgejuvvon maid mii dárbbahit dialogavuogádahkii, ja nubbi dasa lea hábmendokumeanta mii muitala govt dát galgá implementerejuvvot. Mii leat veardidan min plánaid seammásullásaš vuogádagaiguin mat leat ráhkaduvvon eará gielaide, omd. maori, ja ráhkadan prinsihpaid dialogastivremii.

\subsection{Sámegiel cealkagat visl várás}

Mii leat álggahan ságastallamiid olbmuiguin geat leat čállán sámegielat pedagogalaš oahpponeavvuid, ja ulbmil lea maid geavahit sin cealkagiid nu ahte oahppit sáhttet geavahit oahppogirjjiid bargobihtáid maid interaktiivvalaččat interneahtas. Dáinna lágiin spealut leat fálaldahkan sihke skuvllaide ja eará berošteddjiide.

\section{Bargu 2007}

Čakčat 2007 bargaba dušše pedagoga ja datalingvista prošeavttain. Váldobargu lea dalle visl-oassi, masa mii oažžut veahki Syddansk Universitet programmeráriin. Dehálaš oassi barggus lea maid ságastallat visl-cealkagiid giellaoahpalaš analysa čovdosiid sámi lingvisttalaččaiguin ja oahpaheddjiiguin mat barget skuvllas, ja heivehit analysa skuvlla mállii.

Dialogavuogádahkii mii áigut ráhkadit fáttáid, semánttalaš seahtaid, dialogarámmaid ja giellaoahpalaš responssa/bagadallama geavaheddjiide. Dál gerenere min davvisámegielat prográmma gait dohkálaš hámiid, muhto pedagogalaš prográmmii mii dárbbahit generatora mii álo generere dušše ovtta hámi. Mii áigut dan dihte ráhkadit golbma veršuvnna generatoris, okta veršuvnna juohke váldosuopmanii. Nu sáhttá geavaheaddji válljet guđe váldosuopmana son hálida prográmma geavahit. Juohke suopmana ovdanbuktojuvvo "persovnna" bokte geas lea namma ja eallinvásihusat, nu ahte geavaheaddjái orru ahte son gulahallá olbmožiin. Dakkár dialoga bokte geavaheaddji oažžu lunddolaš bagadeami ``ságastallanskihpáris''.
 
\section{Bargu 2008}

Prošeakta lea plánejuvvon bistit jagi 2008 lohppii. Álgojagis mis galgá leat dialogavuogádaga alfaveršuvdna válmmaš geahččaladdamii, ja de joatká bargu buoridit dan, ja de diktit studeanttaid ja earáid geahččaladdat beta-veršuvnna. Dehálaš lea maid oažžut responssa oahpaheddjiin. Seammás álgá min gráfalaš hábmejeaddji hábmetinterneahttasiidduid og sárgut govaid daidda. 

Váldobargu jagi 2008 šaddá dialogasystema - visl-vuođus mii leat jáhkkimis dahkan eanaš barggu jagi 2007. 

\end{document}
