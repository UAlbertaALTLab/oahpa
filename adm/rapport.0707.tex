\documentclass[a4paper,norsk]{article}
\usepackage{babel}
\usepackage{ucs}
\usepackage[utf8x]{inputenc}
\usepackage[T1]{fontenc}
\usepackage{a4wide}

\usepackage{hyperref}
\usepackage{graphics}
\usepackage{url}

\begin{document}


\title{Interaktiv samisk arena på internett: Rapport, juli 2007}


\author{Trond Trosterud ja Lene Antonsen\\
Humanisttalaš fakultehta\\
Romssa Universitehta}

\maketitle


Prosjektet \textit{Interaktiivalaš sámegiel arena interneahtas - hárjehallanbáiki álggahalliidde ja daidda geat juo máhttet olu} har to mål:

\begin{enumerate}
\item Utvikle et sett av programmer for interaktiv trening i grunnleggende kommunikative ferdigheter i samisk. 
\item Utvikle interaktive pedagogiske spel for samisk ved å 
utvide repertoaret av samiske setningar på den interaktive spelplattforma \url{http://visl.sdu.dk}
\end{enumerate}

\section{Arbeid til no}

\subsection{Interaktive pedagogiske spel}

Etter ein del initial planlegging våren 2007, kom prosjektet i gang for fullt først 1.6.2007 da prosjektets pedagog hadde mulighet til å begynne for fullt. Så langt har programmerar, samiskpedagog og datalingvist arbeidd fram to dokument: Eit dokument som inneheld eit inventar over kva som trengst for dialogsystemet, og eit designdokument for korleis dette skal implementerast.   Vi har vurdert tilsvarande system for andre språk, m.a. maori, og sett opp prinsipp for dialaogstyring.

\subsection{Samiske setningar for visl}

Her har vi innleidd diskusjonar med forfattarar av samiske pedagogiske læremiddel, der målet er å inkludere deira setningar, slik at elevane deira kan bruke arbeidsoppgåvene frå læreboka ikkje berre som tradisjonell lekse, men også interaktivt på nettet. Slik vil spela være både et tilbod for skuler og for andre interesserte.

Vi har studert korleis andre språkversjonar har lagt opp sine setningar (jf. t.d. \url{http://www.tekstlab.uio.no/grei/}, og vi har starta med å skrive eigne setningar.

\section{Arbeid i 2007}

Hausten 2007 vil det berre vere samiskpedagog og datalingvist i arbeid med prosjektet. Hovudvekta av arbeidet vil vere på visl-delen, der vi stør oss på programmerarar frå Syddansk Universitet. Et viktig arbeid er å diskutere løsninger for grammatisk analyse av visl-setningene med samiske lingvister og lærare som arbeider i skolen, og tilpasse anlysen til skoleverket sine modeller.

Når det gjeld dialogsystemet vil vi lage temaer, semantiske sett, dialogrammer og respons/vegledning til brukaren når det gjelder grammatiske feil. I dag generer det nordsamiske programmet vårt alle korrekte former, til det pedagogiske programmet treng vi ein generator som alltid generer berre ei form. Vi vil derfor lage tre ulike versjonar av generatoren, ein for kvar hovuddialekt. Slik vil brukarane kunne velje kva for ein hovuddialekt dei ynskjer at programmet skal kommunisere med dei på. Kvar av dialektane blir personleggjort med namn og ei livshistorie, slik at brukaren får kjensla av å kommunisere med ein person. Gjennom ei slik form for dialog vil brukaren få naturleg språklig rettleiiing frå denne ``personen''. 

\section{Arbeid i 2008}

Prosjektet er planlagt å gå ut 2008. I begynnelsen av året skal vi ha en alfaversjon av dialogsystemet klar for utprøving, og så vil arbeidet bestå i å forbetre denne, og så prøve ut en beta-versjon på studenter og andre. Det er viktig også å få respons fra lærere. Samtidig vil vår grafiske formgiver starte arbeidet med å designe og illustrere internettsidene.

Hovudvekta i 2008 vil vere på dialogsystemet, visl-plattforma reknar vi med å ha gjort hovudtyngda av arbeidet på i 2007. 


\end{document}